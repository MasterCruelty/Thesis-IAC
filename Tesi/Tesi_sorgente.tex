%
% Tesi D.S.I. - modello preso da
% Stanford University PhD thesis style -- modifications to the report style
%
%%%%%%%%%%%%%%%%%%%%%%%%%%%%%%%%%%%%%%%%%%%%%%%%%%%%%%%%%%%%%%%%%%%%%%%%%%%
%                                                                         %
%			TESI DOTTORATO                                                   %
%			______________                                                   %
%                                                                         %
%			AUTORE: Elena Pagani                                             %
%                                                                         %
%			Ultima revisione: 7.X.1998                                       %
%           correzioni atrent                                             %
%%%%%%%%%%%%%%%%%%%%%%%%%%%%%%%%%%%%%%%%%%%%%%%%%%%%%%%%%%%%%%%%%%%%%%%%%%%
%
%
\documentclass[a4paper,12pt]{report}
%    \renewcommand{\baselinestretch}{1.6}      % interline spacing
%
% \includeonly{}
%
%			PREAMBOLO
%
\usepackage[a4paper]{geometry}
\usepackage{amssymb,amsmath,amsthm}
\usepackage{graphicx}
\usepackage{url}
\usepackage{hyperref}
\usepackage{epsfig}
\usepackage[italian]{babel}
\usepackage{setspace}
\usepackage{tesi}

% per le accentate
\usepackage[utf8]{inputenc}
%
\newtheorem{myteor}{Teorema}[section]
%
\newenvironment{teor}{\begin{myteor}\sl}{\end{myteor}}
%
%
%			TITOLO
%
\begin{document}
\includegraphics[width=\textwidth]{Logo.jpg}
\title{Realizzazione di una soluzione IAC (Infrastructure As Code) che consenta il rilascio di un'infrastruttura per ambiti DevOps}
\author{\textbf{Roberto Antoniello}}
\dept{Corso di Laurea in Informatica} 
\anno{2022-2023}
\matricola{\textbf{875693}}
\relatore{Prof. Valentina Ciriani}
\correlatore{Mario Petrella}

\beforepreface

\afterpreface
% 
% 
% 
\chapter{Introduzione}
\section{IAC: definizione e vantaggi}
La metodologia IAC(Infrastructure As Code), in italiano "infrastruttura come codice", è una strategia che punta a gestire l'intero ciclo di vita di un'infrastruttura mediante codice senza ulteriore ausilio di altri strumenti. \\ In questo modo è molto più semplice e rapido modificare o eseguire miglioramenti al sistema senza dover ripensare completamente la struttura o stravolgerne le componenti.
Come vedremo nei successivi paragrafi, la potenza di alcune delle tecnologie trattate lungo lo sviluppo di questo progetto risiede proprio nel fatto di poter essere gestite direttamente tramite codice.
\section{Obiettivi del progetto}
Gli obiettivi principali prefissati per questo progetto di tirocinio sono stati 3:
\begin{enumerate}
\item Acquisire competenze in ambito DevOps.
\item Acquisire conoscenze sul ciclo di vita di un'infrastruttura.
\item Costruire un'infrastruttura in grado di essere rilasciata attraverso il cloud e sulla quale fosse possibile rilasciare applicazioni basate su microservizi.
\end{enumerate}
\subsection{Dal metodo tradizionale all'approccio DevOps}
DevOps è una metodologia che punta a ridurre in modo considerevole i tempi di rilascio di nuovo software incorporando nello stesso di team di sviluppo le competenze necessarie per costruire l'infrastruttura adatta al rilascio del software grazie ai concetti di container, microservizi e cloud computing.\\
Mentre con lo sviluppo tradizionale il processo di sviluppo poteva concludersi con una frequenza che poteva anche arrivare a 1 mese, con questa strategia possiamo rilasciare modifiche del software in modo continuativo e automatico nell'arco di 1 ora o anche meno.
\section{Tecnologie utilizzate}
\section{Fasi principali svolte}

\chapter{Studio delle tecnologie coinvolte e analisi dei requisiti}
\section{Tecnologie}
\subsection{Kubernetes}
\subsection{Azure}
\subsection{Terraform}
\subsection{Jenkins}
\section{Requisiti}
\subsection{collegamenti tra le tecnologie}

\chapter{Progettazione dell'infrastruttura}
\section{Casi d'uso}
\section{Definizione dell'infrastruttura}
\subsection{Ulteriori dettagli progettuali}

\chapter{Implementazione in funzione dei disegni progettuali}
\section{Codice Terraform e pipeline dedicata}
\subsection{Definizione via codice}
\subsection{pipeline per la gestione delle modifiche}
\section{Sviluppo pipeline per il rilascio dei micro servizi e modifica di puntamenti nel sorgente}
\subsection{Definizione delle pipeline}
\subsection{Modifica nel codice sorgente}

\chapter{Conclusioni: risultati raggiunti e possibili miglioramenti}
\section{risultati raggiunti al termine}
\section{miglioramenti}

%
%			BIBLIOGRAFIA
%
\begin{thebibliography}{00}
%
%\bibitem{gotti91}
%M. Gotti, I linguaggi specialistici, Firenze, La Nuova %Italia, 1991.
%
%
\end{thebibliography}
% 
\end{document}


 
