%
% Tesi D.S.I. - modello preso da
% Stanford University PhD thesis style -- modifications to the report style
%
%%%%%%%%%%%%%%%%%%%%%%%%%%%%%%%%%%%%%%%%%%%%%%%%%%%%%%%%%%%%%%%%%%%%%%%%%%%
%                                                                         %
%			TESI DOTTORATO                                                   %
%			______________                                                   %
%                                                                         %
%			AUTORE: Elena Pagani                                             %
%                                                                         %
%			Ultima revisione: 7.X.1998                                       %
%           correzioni atrent                                             %
%%%%%%%%%%%%%%%%%%%%%%%%%%%%%%%%%%%%%%%%%%%%%%%%%%%%%%%%%%%%%%%%%%%%%%%%%%%
%
%
\documentclass[a4paper,12pt]{report}
%    \renewcommand{\baselinestretch}{1.6}      % interline spacing
%
% \includeonly{}
%
%			PREAMBOLO
%
\usepackage[a4paper]{geometry}
\usepackage{amssymb,amsmath,amsthm}
\usepackage{graphicx}
\usepackage{url}
\usepackage{hyperref}
\usepackage{epsfig}
\usepackage[italian]{babel}
\usepackage{setspace}
\usepackage{tesi}

% per le accentate
\usepackage[utf8]{inputenc}
%
\newtheorem{myteor}{Teorema}[section]
%
\newenvironment{teor}{\begin{myteor}\sl}{\end{myteor}}
%
%
%			TITOLO
%
\begin{document}
\includegraphics[width=\textwidth]{Logo.jpg}
\title{Realizzazione di una soluzione IAC (Infrastructure As Code) che consenta il rilascio di un'infrastruttura per ambiti DevOps}
\author{\textbf{Roberto Antoniello}}
\dept{Corso di Laurea in Informatica} 
\anno{2022-2023}
\matricola{\textbf{875693}}
\relatore{Prof. Valentina Ciriani}
\correlatore{Dr. Mario Petrella}

\beforepreface

\afterpreface
% 
% 

\chapter{Introduzione}
\section{IAC: definizione e vantaggi}
%
%

%
%			BIBLIOGRAFIA
%
\begin{thebibliography}{00}
%
%\bibitem{gotti91}
%M. Gotti, I linguaggi specialistici, Firenze, La Nuova %Italia, 1991.
%
%
\end{thebibliography}
% 
\end{document}


 
